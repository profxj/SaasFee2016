%%%%%%%%%%%%%%%%%%%% author.tex %%%%%%%%%%%%%%%%%%%%%%%%%%%%%%%%%%%
%
% sample root file for your "contribution" to a contributed volume
%
% Use this file as a template for your own input.
%
%%%%%%%%%%%%%%%% Springer %%%%%%%%%%%%%%%%%%%%%%%%%%%%%%%%%%


% RECOMMENDED %%%%%%%%%%%%%%%%%%%%%%%%%%%%%%%%%%%%%%%%%%%%%%%%%%%
\documentclass[graybox]{svmult}

% choose options for [] as required from the list
% in the Reference Guide

\usepackage{mathptmx}       % selects Times Roman as basic font
\usepackage{helvet}         % selects Helvetica as sans-serif font
\usepackage{courier}        % selects Courier as typewriter font
\usepackage{type1cm}        % activate if the above 3 fonts are
                            % not available on your system
%
\usepackage{makeidx}         % allows index generation
\usepackage{graphicx}        % standard LaTeX graphics tool
                             % when including figure files
\usepackage{multicol}        % used for the two-column index
\usepackage[bottom]{footmisc}% places footnotes at page bottom

% see the list of further useful packages
% in the Reference Guide
\usepackage{amsmath}
\usepackage{amssymb}

\makeindex             % used for the subject index
                       % please use the style svind.ist with
                       % your makeindex program

%%%%%%%%%%%%%%%%%%%%%%%%%%%%%%%%%%%%%%%%%%%%%%%%%%%%%%%%%%%%%%%%%%%%%%%%%%%%%%%%%%%%%%%%%

\newcommand{\HI}{H{\sc I}}
\def\lya{Ly$\alpha$}
\def\mlya{{\rm Ly}\alpha}
\def\ohf{\frac{1}{2}}
\def\ket#1{{|#1\negmedspace>}}
\def\bra#1{{<\negmedspace#1|}}
\def\ltk{\left [ \,}
\def\ltp{\left ( \,}
\def\ltb{\left \{ \,}
\def\rtk{\, \right  ] }
\def\rtp{\, \right  ) }
\def\rtb{\, \right \} }
\def\imp{\, \Rightarrow \>}
\def\sci#1{{\; \times \; 10^{#1}}}
\def\rhf{\frac{3}{2}}
\def\cmma{\;\;\; ,}
\def\perd{\;\;\; .}
\def\smm{\sum\limits}
\def\intl{\int\limits}
\def\rme{{\rm e}}
\newcommand{\mnhi}{N_{\rm HI}}
\newcommand{\nhi}{$\mnhi$}
\def\cm#1{\, {\rm cm^{#1}}}

\begin{document}

\title*{\HI\ Absorption in the Intergalactic Medium}
% Use \titlerunning{Short Title} for an abbreviated version of
% your contribution title if the original one is too long
\author{J. Xavier Prochaska}
% Use \authorrunning{Short Title} for an abbreviated version of
% your contribution title if the original one is too long
\institute{J. Xavier Prochaska \at University of California, 
1156 High St., Santa Cruz, CA 95064 USA \email{xavier@ucolick.org}}
%
% Use the package "url.sty" to avoid
% problems with special characters
% used in your e-mail or web address
%
\maketitle

\abstract*{Each chapter should be preceded by an abstract (10--15 lines long) that summarizes the content. The abstract will appear \textit{online} at \url{www.SpringerLink.com} and be available with unrestricted access. This allows unregistered users to read the abstract as a teaser for the complete chapter. As a general rule the abstracts will not appear in the printed version of your book unless it is the style of your particular book or that of the series to which your book belongs.
Please use the 'starred' version of the new Springer \texttt{abstract} command for typesetting the text of the online abstracts (cf. source file of this chapter template \texttt{abstract}) and include them with the source files of your manuscript. Use the plain \texttt{abstract} command if the abstract is also to appear in the printed version of the book.}

\abstract{Each chapter should be preceded by an abstract (10--15 lines long) that summarizes the content. The abstract will appear \textit{online} at \url{www.SpringerLink.com} and be available with unrestricted access. This allows unregistered users to read the abstract as a teaser for the complete chapter. As a general rule the abstracts will not appear in the printed version of your book unless it is the style of your particular book or that of the series to which your book belongs.\newline\indent
Please use the 'starred' version of the new Springer \texttt{abstract} command for typesetting the text of the online abstracts (cf. source file of this chapter template \texttt{abstract}) and include them with the source files of your manuscript. Use the plain \texttt{abstract} command if the abstract is also to appear in the printed version of the book.}

\section{Historical Introduction}
\label{sec:history}

The discovery of the intergalactic medium (IGM)
was, in essence, precipitated by the discovery of 
quasars in 1963\footnote{There were (failed) attempts
to search for extragalactic gas in 21\,cm absorption
\cite{field59}} \cite{schmidt63}.
It was through spectroscopy of these enigmatic, distant
sources that one could resolve the absorption lines
from gas -- especially \HI\ \lya\ -- 
in the foreground universe.  
Figure~\ref{fig:burb} shows an early example from
\cite{bb+65} taken with the prime-focus spectrograph
on the Shane 120-inch telescope at Lick Observatory.
Even in these early data, one identifies apparently
discrete absorption lines of Hydrogen and heavy
elements establishing the presence of diffuse yet
enriched gas along the sightline.



% For figures use
%
\begin{figure}[b]
\sidecaption
% Use the relevant command for your figure-insertion program
% to insert the figure file.
% For example, with the graphicx style use
\includegraphics[scale=.4]{Figures/burbidge65}
%
% If no graphics program available, insert a blank space i.e. use
%\picplace{5cm}{2cm} % Give the correct figure height and width in cm
%
\caption{Lick spectrum of 3C 191 obtained in February 1666
with the prime-focus spectrograph on the Shane 120-inch
telescope at Lick Observatory.  The comparison lamp spectrum
shown is that of He+Ar.
}
\label{fig:burb}       % Give a unique label
\end{figure}


Spectra like these inspired the first models of
the IGM as discrete absorption lines \cite{bahcall65}
and by inference the first physical insight.
The positive detection of flux at rest wavelengths
shortward of the quasar \lya\ emission line
($\lambda_{\rm rest} < 1215$\AA)
demands a highly ionized IGM.
\cite{gunn65} recognized that a universe with predominantly
neutral hydrogen gas should be opaque to these far-UV
photons and [inferred] -- correctly -- 
that the gas must have a neutral fraction $x_{\rm HI}$ of less 
than 1 part in $10^5$.
As an introduction to the material presented in this Chapter,
we may offer our own rough estimate.
The optical depth of \HI\ \lya\ through
a $\ell = 100$\,kpc portion of the $z=3$ universe
at the mean hydrogen density $\bar n_{\rm H}$ 
is simply

\begin{equation}
\tau(\nu) = \ell \, \bar n_{\rm H} \, x_{\rm HI} \, \sigma_{\mlya}(\nu)
\end{equation}
with $\sigma_{\mlya}$ the \lya\ cross-section.
We estimate the latter assuming Doppler broadening dominates
with a characteristic velocity given by Hubble expansion,
$\Delta v \approx H(z) * \ell \approx 30$\,km/s 
(see Equation~\ref{eqn:Dopp_linecenter}).
Taking a baryonic mass density $\rho_b = 0.0486 \rho_c$
at $z=0$ with $\rho_c$ the critical density and
taking 75\%\ of the baryonic mass as Hydrogen,
we find $\tau(\nu_{jk}) \approx 10^6 x_{\rm HI}$.
Therefore, the positive detection of flux in the 
\lya\ forest demands a highly ionized IGM.

The remainder of the 1960's introduced a series of 
fundamental papers on the astrophysics of absorption-line
analysis, especially by Bachall and his collaborators.
These included the discussion of fundamental diagnostics
of the gas \cite{bahcall66}, 
the application of absorption from the fine-structure 
levels of heavy elements \cite{bahcall67}, and the
assertion that the majority of heavy element absorption
may be associated to the halos of galaxies
\cite{bahcall69a,bahcall69b}.
In a number of respects, the theory had outpaced the
observations.
This held throughout the 1970's, especially for IGM
studies with \HI\ \lya\ although \cite{brown73}
reported the first intergalactic detection of 
\HI\ in 21\,cm absorption.

% For figures use
%
\begin{figure}[b]
\sidecaption
% Use the relevant command for your figure-insertion program
% to insert the figure file.
% For example, with the graphicx style use
\includegraphics[scale=.4]{Figures/young78}
%
% If no graphics program available, insert a blank space i.e. use
%\picplace{5cm}{2cm} % Give the correct figure height and width in cm
%
\caption{\HI\ \lya\ forest spectrum of the quasar PKS 2126--158
obtained by \cite{young78}.  Data like these provided the first
detailed view of the IGM.
}
\label{fig:young}       % Give a unique label
\end{figure}


In the early 1980's, advances in spectroscopic technology
(especially the CCD detector) led to the first high-quality
views of the \HI\ \lya\ forest 
\cite[Figure~\ref{fig:young}]{young78,boks,sargent}.
It was evident from spectra like these that the IGM was
characterized by a stochastic forest of \HI\ absorption
well-described by discrete lines.  [add another line]
This decade also witnessed the first surveys on gas
optically thick at the \HI\ Lyman limit (aka Lyman Limit Systems 
or LLSs; \cite{tytler82})
and on the \HI\ \lya\ absorbers with sufficient column
density to generate damped \lya\ profiles
(aka damped \lya\ systems or DLAs; \cite{wolfe86}).
The field was suddenly awash with data and theory had
now fallen behind.
[Mention Bergeron]
The observers took to developing models of 
`spherical' \HI\ clouds and bull's-eye cartoons to 
describe the gas around galaxies.  
J. Ostriker was the most active theorist on the IGM,
publishing a series of papers on applications of the
IGM including its first phase diagram 
\cite{ostriker83a,ostriker83b,bajtlik87,duncan89}.
But one of his leading models of the day envisioned
"Galaxy formation in an IGM dominated by explostion"
\cite{ostriker80}.


In several respects, the 1990's witnessed the
true maturation of studies the IGM.  Observationally,
the HIRES spectrometer \cite{vogt94} on the 10-m W.M.
Keck telescope fully resolved the IGM at terrific
S/N.  In several respects, these spectra represent
the pinnacle (analogous to Planck on the CMB).
The advance over even the 1980's was profound
as Figure~\ref{fig:HIRES} illustrates.

% For figures use
%
\begin{figure}[b]
\sidecaption
% Use the relevant command for your figure-insertion program
% to insert the figure file.
% For example, with the graphicx style use
\includegraphics[scale=.4]{Figures/Q0636}
%
% If no graphics program available, insert a blank space i.e. use
%\picplace{5cm}{2cm} % Give the correct figure height and width in cm
%
\caption{Comparison of the high quality Palomar spectrum
of Q0636 against the Keck/HIRES data.
}
\label{fig:HIRES}       % Give a unique label
\end{figure}


a new paradigm for the IGM emerged
from hydrodynamic cosmological simulations \cite{miralda96}
and related analytic treatments \cite{huixx}.
The \lya\ clouds were replaced by the Cosmic Web (Figure~\ref{fig:web}),
the filamentary network of dark matter and baryons that 
describes the large-scale structure of a CDM universe.
The \HI\ \lya\ forest traces the undulations in this web
and this so-called 
With this paradigm established, the IGM o
This so-called fluctuating Gunn-Peterson approximation
offers a terrific description of the IGM with sound
analytic underpinnings.


% For figures use
%
\begin{figure}[b]
\sidecaption
% Use the relevant command for your figure-insertion program
% to insert the figure file.
% For example, with the graphicx style use
\includegraphics[scale=.4]{Figures/miralda96}
%
% If no graphics program available, insert a blank space i.e. use
%\picplace{5cm}{2cm} % Give the correct figure height and width in cm
%
\caption{Cosmic web (from \cite{miralda96})
}
\label{fig:web}       % Give a unique label
\end{figure}


Figure~\ref{fig:web_vs_data} compares an early generation
model of the IGM from a hydrodynamic simulation 
against a portion of a Keck/HIRES spectrum.  The
agreement is remarkable and even the expert reader
is challenged to identify which panel is real and
which is simulated.
The cosmic web paradigm is a true triumph of CDM
cosmology and its development ushered in the 
opportunity to leverage IGM observations for fundamental
cosmological [constraints].

% For figures use
%
\begin{figure}[b]
\sidecaption
% Use the relevant command for your figure-insertion program
% to insert the figure file.
% For example, with the graphicx style use
\includegraphics[scale=.4]{Figures/web_vs_HIRES}
%
% If no graphics program available, insert a blank space i.e. use
%\picplace{5cm}{2cm} % Give the correct figure height and width in cm
%
\caption{Cosmic web vs.\ HIRES spectrum
}
\label{fig:web_vs_HIRES}       % Give a unique label
\end{figure}


For the last decade, the observational advances
have stemmed largely form the massive spectroscopic
surveys of SDSS and BOSS.
These have yielded terrific statistical descriptions
of the IGM \cite{PDF} 
across large areas of the sky for BAO \cite{IGM_BAO}.
Large surveys of the IGM have also been comprised
\cite{phw05,pow10,noterdaeme} and
analysis probing the underlying dark matter density
field probed by the IGM have emerged \cite{font}.
In addition, the ongoing discovery of $z>6$ quasars 
and GRBs coupled with high-performance echellette
spectrometers have probed the IGM to the epoch
of \HI\ reionization.  And, a series of increasingly
sensitive UV spectrometers on the
{\it Hubble Space Telescope}
have anchored the results in the modern universe
\cite{penton,tripp,xx}.

This chapter is organized into the following sections:
 (i) the physics of \HI\ \lya\ absorption;
 (ii) key concepts of spectral-line analysis;
 (iii) characterizing the \HI\ \lya\ forest as absorption lines;
 (iv) optically thick \HI\ absorption;
 (v) the damped \lya\ systems;
 (vi) an overview of modern analysis and results.
The focus throughout is observational and the approaches
are largely traditional;  excellent reviews with a greater
emphasis on theory are given by \cite{meiksin0X} and
\cite{mcquinn1X}.  

This Chapter is also supplemented by the lecture notes
and slides presented in SaasFee, and a set of iPython
Notebooks illustrating concepts and providing 
example code for related calculations and modeling.
These supplementary materials are publicaly available
at https://github.com/profxj/SaasFee2016.
Python code relevant to \HI\ \lya\ absorption and 
IGM analysis are packaged as  {\tt linetools}
and {\tt pyigm} on github.

\section{Physics of Lyman Series Absorption}
\label{sec:physics}

The \HI\ \lya\ photon may be emitted following one
of several processes:
 (i) the resonant absorption of a \lya\ photon by atomic hydrogen;
 (ii) as the final emission in the recombination cascade of \HI;
 (iii) following the collisional excitation of atomic hydrogen.
In this Chapter, we concern ourselves with the first process --
\HI\ \lya\ resonant-line scattering -- although the other
processes may play a critical role in \lya\ radiative transfer.

In this section we will
  describe the physics of \HI\ line absorption,
  introduce the concepts of a line-profile, and
  and illustrate the basics of \lya\ absorption lines.
This section is supplemented by the iPython Notebook
"HI\_Lyman\_Series.ipynb" and the "sassfee16\_lymanseries"
Lecture.

\subsection{\HI\ Energy Levels}

We begin with a derivation of the energy levels 
for atomic hydrogen.  Energies in the classic
Rutherford-Bohr model of a Hydrogenic ion with charge
$Ze$ are solved from a standard Hamiltonian ($H^{(0)}$)
with an electrostatic potential

\begin{equation}
H^{(0)} = \frac{-\hbar \nabla^2}{2m} - \frac{Z e^2}{r}
\end{equation}
For energy level $n$, one recovers

\begin{equation}
E_n = -\ohf \mu c^2 \frac{(Z \alpha)^2}{n^2}
\label{eqn:En}
\end{equation}
and the quantum states described by $n$, $\ell$, $m$, $m_s$
or $\ket{n \ell m m_s}$ 
are degenerate in $\ell, m, m_s$
because our Hamiltonian is rotationally invariant.
In Equation~\ref{eqn:En}, we have
the fine-structure constant 
$\alpha \equiv e^2/\hbar c \approx 1/137$ and
the reduced mass $\mu = \frac{m_e (Z m_p)}{m_e + Z m_p}$.
For Hydrogen, $\mu \approx 0.999 m_e$. 

From $E_n$, we may evaluate the wavelengths for the Lyman
series, transitions from/to the ground state $E_1$, as

\begin{equation}
\lambda_{rest,n} = \frac{hc}{E_n - E_1}
\label{eqn:lrest}
\end{equation}
Table~\ref{tbl:wrest} lists the calculated values 
from Equations~\ref{eqn:En} and \ref{eqn:lrest}
for $\approx 20$ Lyman series lines, compared against
empirical measurements.  One identifies a systematic
offset of $\delta\lambda \approx 0.015$\AA\ between
the Rutherford-Bohr energies and experiment.
These result from perturbations to the 
standard Hamiltonian that we now consider.

\begin{table}[ht]
\begin{center}
\caption{Lyman Series Lines \label{tab:energies}}
\begin{tabular}{cccccc}
\hline
Transition & $n$ & $E_n - E_1$ & $\lambda_{rest}$ & $\lambda_{exp}$ \\
& & (eV) & (\AA) & (\AA) \\
\hline
Ly$\alpha$&           2&       10.200000&       1215.6845& 1215.6701 \\
Ly$\beta$&            3&       12.088889&       1025.7338& 1025.7223\\
Ly$\gamma$&           4&       12.750000&       972.54759&  972.5368\\
Ly$\delta$&           5&       13.056000&       949.75351&  949.7431\\
Ly$\epsilon$&         6&       13.222223&       937.81375&  937.8035\\
Ly6&                  7&       13.322449&       930.75844&  930.7483\\
Ly7&                  8&       13.387500&       926.23580&  926.2257\\
Ly8&                  9&       13.432099&       923.16041&  923.1504\\
Ly9&                  10&       13.464000&       920.97310& 920.9631\\
Ly10&                 11&       13.487604&       919.36139& 919.3514\\
Ly11&                 12&       13.505556&       918.13933& 918.1294\\
Ly12&                 13&       13.519527&       917.19053& 917.1806\\
Ly13&                 14&       13.530613&       916.43908& 916.429\\
Ly14&                 15&       13.539556&       915.83374& 915.824\\
Ly15&                 16&       13.546875&       915.33891& 915.329\\
Ly16&                 17&       13.552942&       914.92921& 914.919\\
Ly17&                 18&       13.558025&       914.58616& 914.576\\
Ly18&                 19&       13.562327&       914.29604& 914.286\\
Ly19&                 20&       13.566000&       914.04849& 914.039\\
Ly$\infty$&         $\infty$&   13.6&            912.6    & \\
\hline
\end{tabular}
\end{center}
\end{table}


There are two perturbations with energies that scale
as $\alpha^4$, the next term $E_n$:
(1) spin-orbit coupling and
(2) the first expansion of the relativistic kinetic energy.
Classically, the spin-orbit coupling is described as 
a magnetic dipole interaction between the spin of the electron
and the orbit of the nucleus, i.e.\ 
the e$^{-}$ observes a magnetic field due
to the current driven by the nucleus:

\begin{equation}
\vec B = - \frac{1}{c} \vec v \times \vec 
E = \frac{1}{m_e c r} \; \vec \ell \;
\frac{d\phi}{dr}
\end{equation}
with energy

\begin{equation}
E = - \vec \mu_s \cdot \vec B  \quad\quad \text{with} \;\;\; 
\vec \mu_s = -\frac{e g \vec s}{2 m_e c}
\end{equation}
where $g \approx 2$ for an electron.
The proper Hamiltonian for the perturbation is:

\begin{equation}
H_{SO} = \frac{1}{2 m_e^2 c^2} \; \vec S \cdot \vec L \; \frac{1}{r} \;
\frac{d\phi}{dr}
\end{equation}
and the standard treatment is to 
identify an operator which commutes with 
$H^{(0)}$ and $H_{SO}$, and also
uniquely identifies the degenerate states.
We choose  $\vec J \equiv \vec L + \vec S$
recognizing

\begin{equation*}
\vec L \cdot \vec S = \ohf \ltp |\vec J|^2 - |\vec L|^2 - |\vec S|^2 \rtp
\end{equation*}
which yields energies

\begin{equation}
E_{SO} = \, <H_{SO}> \, = \ohf C \ltk j(j+1) - \ell(\ell+1) - s(s+1) \rtk
\end{equation}
with $C$ a constant.
For fixed $\vec L$ and $\vec S$ (i.e.\ splitting within a level), 
e.g.\ 2P, we have $\Delta E_{SO} = E_{J+1} - E_J = C (j+1)$.
For our Hydrogenic ion,
$\phi = Z e^2/r \imp d \phi/dr = -Z e^2/r^2$ and

\begin{equation}
H_{SO} = \frac{Z e^2}{2 m_e^2 c^2} \frac{1}{r^3} \vec L \cdot \vec S  \;\; .
\end{equation}
Following standard perturbation 
theory\footnote{See the Lecture notes for an expanded derivation.},
we find

\begin{equation}
<H_{SO}> = -E_n \frac{Z^2 \alpha^2}{2n} \frac{[j(j+1) - 
\ell(\ell+1) - s(s+1)]}{\ell(\ell+\ohf)(\ell+1)}
\end{equation}
with $E_n$ given by Equation~\ref{eqn:En}.
As advertised, spin-orbit coupling is 4th order in $\alpha$
and we now have explicit energy dependence on $j,\ell$ and $s$.
For the Hydrogen $n=2$ levels ($Z=1; n=2; \ell = 0,1; j= 1/2, 3/2$),
we find

\begin{equation}
<H_{SO}> = 
  \begin{cases}
	0 &  2 ^2{\rm S}_\ohf \quad\quad (\ell = 0; j=0) \\
     \frac{mc^2 \alpha^4}{96} &  2 ^2{\rm P}_{3/2} \quad\quad (\ell=1; j=3/2) \\
     \frac{-mc^2 \alpha^4}{48} &  2 ^2{\rm P}_{1/2} \quad\quad (\ell=1; j=1/2) \\
  \end{cases}
\end{equation}
giving a 2P$_{3/2}$ - 2P$_{1/2}$ splitting of $4.5 \sci{-5}$~eV
or $\Delta v \approx \Delta E / cE \approx 1$~km/s.

To derive the Relativistic correction to order $\alpha^4$,
we expand the K.E. to the next term in $v^2/c^2$ from the Lagrangian

\begin{equation}
\text{K.E.} = \frac{p^2}{2m} \ltp 1 - \frac{1}{4} \frac{v^2}{c^2} \rtp
\end{equation}
This gives a relativistic perturbation 

\begin{equation}
H_{rel} = -\frac{1}{2mc^2} \ltp \frac{p^2}{2m} \rtp^2
\end{equation}
that has no spin dependence (spherically symmetric)
such that $[H_{rel}, L^2] = [H_{rel}, L] = 0$
and the standard $\ket{n \ell m m_s}$ diagonalize $H_{rel}$.
If we recognize that

\begin{equation*}
H_{rel} = -\frac{1}{2mc^2} \ltp H^{(0)} - V^{(0)} \rtp^2
\end{equation*}
with $V^{(0)} = -Z e^2/r$, it is straightforward to
compute the energies

\begin{equation}
<H_{rel}> = -E_n \frac{Z^2 \alpha^2}{n} \ltp \frac{3}{4n} - \frac{1}{\ell + \ohf} \rtp
\end{equation}

Combining this result with spin-orbit coupling,
we recover (truly remarkably), 

\begin{equation}
<H_{SO}> + <H_{rel}> \, = E_n \frac{Z^2 \alpha^2}{n} \, 
\ltk \frac{1}{j+\ohf} - \frac{3}{4n} \rtk
\end{equation}
Altogether, the $\alpha^4$ term for $E_n$ has
no explicit $\ell$ dependence, nor any $s$ dependence.
Furthermore, higher $j$ implies higher energy 
following the 3rd Hund's rule.
The energy shifts from $E_n$ for Hydrogen are then

\begin{equation}
\Delta E_{nj} = -7.25 \sci{-4} {\rm eV} \; \frac{1}{n^3} 
   \ltk \frac{1}{j+\ohf} - \frac{3}{4n} \rtk
\end{equation}
which for the $n=1,2$ states of Hydrogen evaluate to
the results in Table~\ref{tab:Enshift}.

\begin{table}
\label{tab:Enshift}
\caption{Perturbations to the $n=1,2$ levels of Hydrogen}
\begin{center}
\begin{tabular}{lccc}
\hline
State & $n$ & $j$ & $<H_{SO}> + <H_{rel}>$ \\
\hline
1\,$^2$S$_{\ohf}$ & 1 & $\ohf$ & $-1.8 \sci{-4}$~eV \\
2\,$^2$S$_{\ohf}$, 2\,$^2$P$_\ohf$ & 2 & $\ohf$ & $-5.7 \sci{-5}$~eV \\
2\,$^2$P$_\rhf$ & 2 & $\rhf$ & $-1.1 \sci{-5}$~eV \\
\hline
\end{tabular}
\end{center}
\end{table}

The splitting of the $n=2$ level implies that
Ly$\alpha$ (1S-2P) is a doublet with $\approx 1$km/s separation
which is generally too small to resolve observationally
but could be important for radiative transfer treatments.

Returning to Table~\ref{tab:En}, the implied
shift from the Rutherford-Bohr energies
$\delta\lambda/\lambda \sim \delta E/E$
is calculated using $\delta E$ from Table~\ref{tab:Enshift}
and accounting for the relative degeneracy of the 
2\,$^2$P$_\ohf$, 2\,$^2$P$_\rhf$ states
yields

\begin{align}
\delta E &= \frac{1}{3} \ltk 2\delta E_{1S \to 2P_\rhf} + \delta E_{1S \to 2P_\ohf}
   \rtk \\
         &= 1.55 \sci{-4}~{\rm eV}
\end{align}
or
\begin{equation}
\Delta \lambda = -\frac{\delta E}{E} \times 1215.68 {\rm \AA} = -0.0138 {\rm \AA} \perd
\end{equation}
Voila!

%%%%%%%%%%%%%%%%%%%%%%%%%%%%%%%%%%%%%%%%%%%%%%%%%%%%%%%%
\subsection{The Line Profile}

We now derive the line profile of \HI\ Lyman series absorption
which expresses the energy dependence of the
photon cross-section.   This results from two physical
effects: 
(1) the quantum mechanical coupling of the energy
levels described in the previous sub-section; 
(2) Doppler broadening from the kinetic motions of the gas.
We reserve a discussion of the observed
line-profile related to the instrument to the next section.

We express the opacity $\kappa_{\rm jk}(\nu)$
of a gas with number
density $n_j$ in state $j$ as:

\begin{equation} 
\kappa_{\rm jk}(\nu) = n_j \sigma_{jk}(\nu)
\end{equation} 
with $\sigma_\nu$ the photon cross-section 
at frequency $\nu$ for a transition to state $k$.
Separate the frequency dependence by introducing
the line-profile $\phi(\nu)$

\begin{equation}
\sigma(\nu) = \sigma_{jk} \phi(\nu)
\end{equation}
with $\sigma_{jk}$ the integrated cross-section over all frequencies
and $\phi_\nu \, d\nu$ reflects the probability an atom will absorb
a photon with energy in $\nu, \nu + d\nu$.

A naive guess for $\phi(\nu)$ is the Delta function,
i.e.\ the transitions occur only at the exact energies
splitting the energy levels from the previous subsection:

\begin{equation}
\phi_\delta(\nu) = \delta \ltp \nu - \nu_{jk} \rtp
\end{equation}
Quantum mechanically, however, the excited $n=2$
state has a half-life $\tau_\ohf$ 
given by the Spontaneous emission coefficient,
$\tau_\ohf = 1/A_{\rm jk}$.
This finite lifetime implies a finite width
$\Delta E$ to the energy level which may be estimated
from the Heisenberg Uncertainty principle

\begin{equation}
\Delta E \sim \frac{\hbar}{\Delta t} \sim \hbar A
\end{equation}
Define $W_{jk}(E)$ as the quantum mechanical
probability of a transition occurring 
between states $j$ and $k$ with energy $E$
and $W_j(E)$ 
as the probability of state $j$ being characterized
by the energy interval $(E_j, E_j+dE_j)$.
A standard Quantum mechanical treatment shows that
$W_j(E)$ has a Lorentzian shape (aka the Breit-Wigner
profile): 

\begin{equation}
W_j(E_j) dE_j = \frac{\gamma_j dE_j/h}{(2\pi/h)^2 \ltk E_j - 
<E_j>\rtk^2 + (\gamma_j/2)^2}
\end{equation}
with  

\begin{equation}
\gamma_j \equiv \smm_{i<j} A_{ij}
\end{equation}


For a coupling between only two states $j,k$, one
derives $W_{jk}$ by convolving $W_j$ and $W_k$: 

\begin{equation}
W_{jk}(E) dE = \frac{\ltk \gamma_j + \gamma_k \rtk dE/h}{
(2\pi/h)^2 \ltk E - E_{jk}\rtk^2 + ([\gamma_j + \gamma_k]/2)^2}
\label{eqn:Wjk}
\end{equation}
We reemphasize that this calculation was restricted 
to the $j$ and $k$ states whereas a proper 
calculation needs to consider the coupling of all the 
energy levels (see below).  For our Lyman series lines we
note that $\gamma_j=0$ as, by definition, there are 
no energy levels below the ground state.

From Equation~\ref{eqn:Wjk}, we introduce
the Natural line-profile normalized to have 
unit integral value in frequency,

\begin{equation}
\phi_N(\nu) = \frac{1}{\pi} 
\ltk \frac{(\gamma_j + \gamma_k)/4\pi}{(\nu - \nu_{jk})^2
+ (\gamma_j + \gamma_k)^2 / (4\pi)^2} \rtk \perd
\end{equation}
giving (at last)

\begin{equation}
\sigma(\nu) = \sigma_{jk} \phi_N(\nu)
\end{equation}
As illustrated in the Notebook, 
the "wings" of $\phi_N(\nu)$ are 
$\approx 10$ orders of magnitude down 
from line-center.  Remarkably
these are very important for \lya.

It is standard practice to express the normalization
$\sigma_{jk}$ in terms of the oscillator strength $f_{jk}$
which is either measured empirically (preferred)
or estimated theoretically

\begin{equation}
\sigma_{jk} = \frac{\pi e^2}{m_e c} f_{jk}
\end{equation}

Our final expression becomes

\begin{equation}
\sigma_\nu = \frac{\pi e^2}{m_e c} f_{jk} 
\ltk \frac{(\gamma_j + \gamma_k)/4\pi^2}{(\nu - \nu_{jk})^2
+ (\gamma_j + \gamma_k)^2 / (4\pi)^2} \rtk
\end{equation}

[NEED to GIVE $A$ for \lya\ somewhere]

Expressing the FWHM of the line-profile as 
the frequency width where $\sigma(\nu)/\sigma(\nu)_{max} = 1/2$,
we have $\Delta\nu_{\rm FWHM} = \pm \frac{\gamma_j + \gamma_k}{4\pi}$
which very nearly matches our estimate from the
Uncertainty Principle!
For \HI\ \lya\ with $\gamma_1=0, \gamma_2=A_{21}$,
we find the FWHM in velocity to be 

\begin{equation}
\Delta v_{\rm FWHM} = c \Delta\nu_{\rm FWHM}/\nu
\approx 1.5\sci{-2} \, {\rm km/s}  
\label{eqn:vFWHM}
\end{equation}
For astrophysical purposes,
this nearly is a delta function.

In an astrophysical event, each 
atom in a gas has its own motion 
which spreads the line without changing the total 
amount of absorption.  
This Doppler effect, to lowest order in $v/c$, is

\begin{equation}
\Delta\nu = \nu - \nu_{jk} = \nu_{jk} \frac{v}{c}
\end{equation}
Assuming first that the gas motions are characterized
soley by $T$ (i.e.\ no turbulence),
we adopt a Maxwellian distribution for particles of mass $m_A$
giving a profile function:

\begin{align}
\phi_D(\nu) &= \frac{1}{\Delta \nu_D \sqrt{\pi}} 
\exp \ltk - \frac{(\nu - \nu_{jk})^2}{\Delta \nu_D^2} \rtk \\
{\rm with} \; \Delta \nu_D &\equiv \frac{\nu_{jk}}{c} \sqrt{\frac{2kT}{mA}}
\end{align}
At line-center ($\nu = \nu_{jk}$),
the cross-section (neglecting stimulated emission) is

\begin{equation}
(\sigma_\nu^D)_{\rm max} = \sigma_{jk} \phi_D(\nu_{jk})  
	= \frac{\pi e^2}{mc} f_{jk} \frac{1}{\Delta \nu_D \sqrt{\pi}}
\label{eqn:Dlinecenter}
\end{equation}
This cross-section is several orders of magnitude
lower than $(\sigma_\nu^N)_{\rm max}$ as the absorption has
been `spread' over a velocity interval several orders of
magnitude larger than the width estimated by Equation~\ref{eqn:vFWHM}.

We can generalize the profile to include random, turbulent
motions (characterized by $\xi$)
by modifying the Doppler width,

\begin{equation}
\Delta\nu_D = \frac{\nu_{jk}}{c} \ltp \frac{2kT}{m_A} + 
\xi^2 \rtp^\ohf
\end{equation}
Expressing the line-profile in velocity space, we introduce
the Doppler parameter

\begin{equation}
b \equiv \sqrt{\frac{2kT}{m_A} + \xi^2}
\end{equation}
and the velocity line-profile for Doppler motions is

\begin{equation}
\phi_D(v) = \frac{1}{b\sqrt{\pi}} \, \exp \ltk -\frac{v^2}{b^2} \rtk \perd
\label{eqn:phi_Dv}
\end{equation}
See the Notebook for a series of examples illustrating this
profile in comparison to the Natural profile.


Generally, the cross-section has contributions from both
Natural and Doppler broadening,
with Doppler broadening dominating the line-center 
and the Lorentzian of Natural broadening dominates the wings.
The overall profile is a convolution of the two terms,

\begin{equation}
\phi_V(\nu) = \frac{\gamma}{4\pi} \intl_{-\infty}^\infty
\frac{ \ltp \frac{m}{2\pi kT} \rtp^\ohf \, \exp \ltp - \frac{mv^2}{2kT} \rtp}
{\ltp \nu-\nu_{jk}-\nu_{jk}v/c \rtp^2 + \ltp \gamma/4\pi \rtp^2} \, dv
\end{equation}
and one is inspired to introduce the Voigt function

\begin{equation}
H(a,u) = \frac{a}{\pi} \intl_{-\infty}^\infty 
\frac{\rme^{-y^2} \, dy}{a^2 + (u-y)^2}
\end{equation}
where we identify

\begin{align}
a &\equiv \frac{\gamma}{4\pi \Delta\nu_D} \\
u &\equiv \frac{\nu-\nu_{jk}}{\Delta\nu_D}
\end{align}
Altogether, we have 

\begin{equation}
\phi_V(\nu) = \frac{H(a,u)}{\Delta\nu_D \sqrt{\pi}}
\end{equation}
which has no analytic solution.
For speed, one often relies on look-up tables.
In Python, the real part of scipy.special.wofz is both accurate 
and fast (see Voigt documentation in the {\tt linetools} package).

The Lorentzian profile from Natural broadening is only an approximation
because it ignores the coupling between states other than $j,k$.
Scattering is a second-order quantum mechanical process: annihilation of 
one photon and the creation of a scattered photon.
A proper treatment requires second-order time dependent perturbation
theory and requires one to sum over all bound-states and integrate over
all continuum-state contributions.
\cite{lee03} have calculated a series expansion of the corrections
to the Voigt profile and we refer the reader to his paper for a
more detailed presentation.  Here, we report the first term
in the series for \HI\ \lya:

\begin{equation}
\sigma_\nu = \sigma_{\rm T} \ltp \frac{f_{jk}}{2} \rtp^2 
\ltp \frac{\nu_{jk}}{\delta\nu} \rtp^2
\ltk 1 - 1.792 \frac{\delta\nu}{\nu_{jk}} \rtk  \perd
\end{equation}
with $\sigma_T$ the Thompson cross-section.
It is evident that the correction is not symmetric about line-center.
This leads to an asymmetry which
shifts the measured line-center if the gas opacity
is very high (e.g.\ $\mnhi > 10^{21.7} \cm{-2}$).

\subsection{Optical Depth ($\tau_\nu$) and Column Density ($N$)}
\label{subsec:tauN}

We now introduce two quantities central to absorption-line analysis.
First, the optical depth $\tau_\nu$ which is defined as the integrated
opacity along a sightline.  In differential form
$d\tau(\nu) = -\kappa(\nu) ds$ implying

\begin{equation}
\tau(\nu) = \sigma(\nu) \int n_j ds
\end{equation}
which gives an explicit frequency dependence related to the line-profiles
of the previous sub-section.

We define the second term as the column density $N_j$,

\begin{equation}
N_j \equiv \int n_j ds
\end{equation}
which has units of cm$^{-2}$ and is akin to a surface density
($n_j \to \rho$; $N_j \to \Sigma$).
As an example, consider the column density of 
of O$_2$ through 1\,m of air.  
With $\rho_{\rm O_2} = 1.492$\,g/L,
$N_{\rm O_2} = 3\sci{21} \cm{-2}$.

A quantity of observational interest is the 
optical depth at line-center $\tau_0$ of a gas with $N_j$
and Doppler parameter $b$.
At line-center ($\nu = \nu_{jk})$, our line-profile is 
dominated by Doppler motions
$\phi_V(\nu_{jk}) \approx \phi_D (\nu_{jk})$ and from 
Equation~\ref{eqn:phi_Dv}, we recover
\begin{equation}
\tau_0 =  \frac{\sqrt{\pi} e^2}{m_e c} \, \frac{N_j \lambda_{jk} f_{jk}}{b} \perd
\end{equation}
For Ly$\alpha$, with $b$ expressed in km/s:
\begin{equation}
\tau_0^{\rm Ly\alpha} =  7.6 \sci{-13} \, 
\frac{N \, [\rm cm^{-2}]}{b \, [\rm km/s]}
\label{eqn:tau0}
\end{equation}

\subsection{Idealized Absorption Lines}

Without derivation (see the Lecture notes),
we express the radiative transfer for a source with
intensity $I^*_\nu$ through a medium with
optical depth $\tau_\nu$ simply as

\begin{equation}
I_\nu = I_\nu^* \rme^{-\tau_\nu}
\end{equation}
Therefore, we may consider the formation of absorption lines as the 
simple integration of the optical depth.

% For figures use
%
\begin{figure}[b]
%\sidecaption
% Use the relevant command for your figure-insertion program
% to insert the figure file.
% For example, with the graphicx style use
\includegraphics[scale=.4]{Figures/fig_lya_lines}
%
% If no graphics program available, insert a blank space i.e. use
%\picplace{5cm}{2cm} % Give the correct figure height and width in cm
%
\caption{\lya\ lines
}
\label{fig:lyalines}       % Give a unique label
\end{figure}


In Figure~\ref{fig:lyalines}, we show a series of idealized \HI\ \lya\
lines for a range of \HI\ column densities and Doppler parameters
(\nhi, $b$).  These illustrate the shapes of the line-profiles, here
dominated by Doppler broadening, and the varying optical depth
at line-center.  These are plotted in velocity space, taking
$\delta v = \delta \lambda / \lambda_0$ with 
$\delta v=0$\,km/s corresponding to the line-center.  
In Figure~\ref{fig:dla_compare}
we contrast these lines, which have $\tau_0 \approx 1$, with 
an \HI\ absorption line with very high \nhi\ and correspondingly
high $\tau_0$.  Here the line profile is entirely determined
by Natural broadening.

% For figures use
%
\begin{figure}[b]
%\sidecaption
% Use the relevant command for your figure-insertion program
% to insert the figure file.
% For example, with the graphicx style use
\includegraphics[scale=.4]{Figures/fig_dla_compare}
%
% If no graphics program available, insert a blank space i.e. use
%\picplace{5cm}{2cm} % Give the correct figure height and width in cm
%
\caption{DLA vs.\ \lya\ lines
}
\label{fig:lyalines}       % Give a unique label
\end{figure}

\subsection{Equivalent Width}
\label{subsec:EW}

The equivalent width for absorption 
$W_\lambda$ is a gross measure of the 
flux absorbed (scattered) by the gas cloud.
Strictly, $W_\lambda$ is the convolution of the 
optical depth with the line profile.  
And although it is primarily an observational quantity,
its value is {\it independent} of the instrument profile
and it does depend on physical properties of the
absorbing gas.


% For figures use
%
\begin{figure}[b]
\sidecaption
% Use the relevant command for your figure-insertion program
% to insert the figure file.
% For example, with the graphicx style use
\includegraphics[scale=.4]{Figures/example_ew.pdf}
%
% If no graphics program available, insert a blank space i.e. use
%\picplace{5cm}{2cm} % Give the correct figure height and width in cm
%
\caption{Box car description of EW
}
\label{fig:EWbox}       % Give a unique label
\end{figure}

The value, expressed in \AA, may be
visualized as the width of a box-car profile
that matches the absorbed flux in a normalized 
spectrum, e.g. Figure~\ref{fig:EWbox}.
Analytically, we define

\begin{equation}
W_\lambda = \intl_0^\infty \ltk 1 - \frac{I_\nu}{I_\nu^*} \rtk \, d\lambda
\end{equation}
Substituting our simple radiative transfer equation, this gives

\begin{equation}
W_\lambda = \intl_0^\infty \ltk 1 - \exp(-\tau_\lambda) \rtk \, d\lambda
\label{eqn:EW}
\end{equation}
One may contrast this equation with analysis of stellar 
atmospheres which has
a very different radiative transfer equation.

\subsection{Curve of Growth for \HI\ \lya}
\label{subsec:COGlya}

It is valuable to develop an intuition 
of the relation bbetween equivalent width and the
physical properites of a gas ($N,b$).  This 
relationship is generally referred to as the
curve-of-growth (COG) and may be inverted to
constrain $N,b$ from $W_\lambda$ measurements.
The COG also nicely describes the transition from a 
Doppler-dominated line to a naturally-broadened line.

On typically considers three regimes for the COG
which depend on the central optical depth of the
absorption line $\tau_0$.
In the Weak limit ($\tau_0 \ll 1$), Natural 
broadening is negligible and Doppler broadening
dominates,

\begin{equation}
\tau_\nu = \frac{\pi e^2}{m_e c} \lambda f_{jk} N_j \phi_D(\nu) 
\end{equation}
With $\tau_0$ small, Equation~\ref{eqn:EW} reduces to
\begin{align}
W_\lambda &= \frac{\lambda^2}{c} \int \ltk 1 - \exp(-\tau_\nu) \rtk d\nu \\
  & \approx \frac{\lambda^2}{c} \intl_0^\infty \tau_\nu \, d\nu \\
  & = \frac{\lambda^2}{c} \frac{\pi e^2}{m_e c} f_{jk} N_j 
\end{align}
revealing a linear relationship between $W_\lambda$ and $N_j$
and no dependence on $b$.
The Weak limit, therefore, is also referred
to as the `linear' portion of the COG.
Evaluating for \lya, we find
\begin{equation}
W_\lambda \approx (0.1 \, {\rm \AA}) \, \frac{N_{\rm HI}}{1.83 \sci{13} \cm{-2}} 
\end{equation}
Our estimate of $\tau_0$ (Equation~\ref{eqn:tau0})
requires $\mnhi \ll 10^{14} \cm{-2}$ for $\tau_0 \ll 1$,
which also implies $W_\lambda \ll 1$\AA.
This column density is many orders of magnitude less than 
that typcial of the Galactic ISM.  It implies a gas that 
is extremely diffuse and, likely, highly ionized.  These
are the physical conditions of the IGM.



% For figures use
%
\begin{figure}[b]
\sidecaption
% Use the relevant command for your figure-insertion program
% to insert the figure file.
% For example, with the graphicx style use
\includegraphics[scale=.4]{Figures/ew_sat_fill}
%
% If no graphics program available, insert a blank space i.e. use
%\picplace{5cm}{2cm} % Give the correct figure height and width in cm
%
\caption{Saturated \lya\ line in the Strong regime of the
COG
}
\label{fig:EWstrong}       % Give a unique label
\end{figure}



The Strong line limit of the COG refers to $\tau_0 \gtrsim 1$
and we may still ignore Natural broadening.
In this regime, the light is strongly absorbed at line-center
and the equivalent width is well described by the width
of the line (i.e.\ it is nearly approximated as a `box';
Figure~\ref{fig:EWstrong}).
Expressing the frequency dependence of the optical
depth as $\tau_x = \tau_0 \rme^{-x^2}$, with
$x \equiv \Delta \nu / \Delta \nu_D$, we may estimate the 
line width by 
considering the value of $x$ that gives $\tau_x = 1$.
Trivially, we find $x_1 = \sqrt{\ln \tau_0}$ and therefore

\begin{equation}
W_\lambda \approx 2 x_1 \approx 2 \sqrt{\ln \tau_0}
\label{eqn:EWstrong}
\end{equation}
in the Strong regime, also known as the saturated limit.
To change $W_\lambda$ in this saturated
limit, we need to increase $\tau_0$ immensely
and likewise $N$, i.e.\ $W_\lambda \propto (\ln N)^\ohf$.
Whereas $W_\lambda$ is insensitive to
$\tau_0$, it is sensitive to the internal structure of the cloud
$W_\lambda \propto b$.

Lastly, there is the Damping regime with $\tau_0 \gg 1$
and where Natural broadening dominates.
Now the optical depth is given by the Lorentzian profile

\begin{equation}
\tau_x \approx \frac{\tau_0 A}{\sqrt{\pi}} \frac{1}{x^2}
\end{equation}
and the width (estimated from $\tau_x = 1$) is
$x_1 = \ltk \tau_0 A \rtk^\ohf$.
Therefore, $W_\lambda \approx 2x_1 \propto N^\ohf$
or formally

\begin{equation}
\frac{W_\lambda}{\lambda} \approx \frac{2}{c} \ltk \lambda^2 N_j 
\frac{\pi e^2}{m_e c} f_{jk} A \rtk^\ohf
\end{equation}
In the Damping regime, the equivalent width scales
as $\sqrt{N}$ with no dependence on the Doppler parameter
as the core is fully saturated.

% For figures use
%
\begin{figure}[b]
\sidecaption
% Use the relevant command for your figure-insertion program
% to insert the figure file.
% For example, with the graphicx style use
\includegraphics[scale=.4]{Figures/fig_cog}
%
% If no graphics program available, insert a blank space i.e. use
%\picplace{5cm}{2cm} % Give the correct figure height and width in cm
%
\caption{COG
}
\label{fig:COG}       % Give a unique label
\end{figure}


Figure~\ref{fig:COG} shows the COG for a single
\HI\ \lya\ line with varying \nhi\ and Doppler parameter.
The three COG regimes are well-described.  We stress
that only a small portion of the Weak limit ($W_\lambda \ll 1$\AA)
permits actual detections with modern spectrometers
and that the Damping regime is limited to gas with
galactic surface densities.
This leaves approximately 6 orders of magnitude in 
\nhi\ in the Strong (saturated) regime where observations
of the equivalent width for
\lya\ alone offer a weak constrain on the gas column
density.

\subsection{Curve of Growth for the Lyman Series}

Another application of the COG is to evaluate the
absorption from a series of transitions from as 
single ion in a single `cloud' of gas.
Such analysis may enable one to derive more precisely
the physical parameters ($N,b$) of the absorbing gas.
Recalling that 

\begin{equation}
\tau_0 = \frac{\sqrt{\pi} e^2}{m_e c} \frac{\lambda_{jk} f_{jk} N_j}{b} \cmma
\end{equation}
a single hydrogen gas cloud with fixed $N,b$ will
show decreasing $\tau_0$ in increasing terms of the
Lyman series. Therefore, the Lyman series absorption 
generates a COG that may span several regimes.

% For figures use
%
\begin{figure}[b]
\sidecaption
% Use the relevant command for your figure-insertion program
% to insert the figure file.
% For example, with the graphicx style use
\includegraphics[scale=.4]{Figures/fig_excog}
%
% If no graphics program available, insert a blank space i.e. use
%\picplace{5cm}{2cm} % Give the correct figure height and width in cm
%
\caption{COG fit
}
\label{fig:COGfit}       % Give a unique label
\end{figure}


The standard analysis is to fit COG curves to a
series of $W_\lambda$ measurements to constrain $N,b$
as illustrated in Figure~\ref{fig:COGfit}.
Another example and related code are provided in the
Notebook.


% For figures use
%
\begin{figure}[b]
\sidecaption
% Use the relevant command for your figure-insertion program
% to insert the figure file.
% For example, with the graphicx style use
\includegraphics[scale=.4]{Figures/web_vs_HIRES}
%
% If no graphics program available, insert a blank space i.e. use
%\picplace{5cm}{2cm} % Give the correct figure height and width in cm
%
\caption{Cosmic web vs.\ HIRES spectrum
}
\label{fig:web_vs_HIRES}       % Give a unique label
\end{figure}


% Always give a unique label
% and use \ref{<label>} for cross-references
% and \cite{<label>} for bibliographic references
% use \sectionmark{}
% to alter or adjust the section heading in the running head
Instead of simply listing headings of different levels we recommend to
let every heading be followed by at least a short passage of text.
Further on please use the \LaTeX\ automatism for all your
cross-references and citations.

Please note that the first line of text that follows a heading is not indented, whereas the first lines of all subsequent paragraphs are.

Use the standard \verb|equation| environment to typeset your equations, e.g.
%
\begin{equation}
a \times b = c\;,
\end{equation}
%
however, for multiline equations we recommend to use the \verb|eqnarray| environment\footnote{In physics texts please activate the class option \texttt{vecphys} to depict your vectors in \textbf{\itshape boldface-italic} type - as is customary for a wide range of physical subjects}.
\begin{eqnarray}
a \times b = c \nonumber\\
\vec{a} \cdot \vec{b}=\vec{c}
\label{eq:01}
\end{eqnarray}

\subsection{Subsection Heading}
\label{subsec:2}
Instead of simply listing headings of different levels we recommend to
let every heading be followed by at least a short passage of text.
Further on please use the \LaTeX\ automatism for all your
cross-references\index{cross-references} and citations\index{citations}
as has already been described in Sect.~\ref{sec:2}.

\begin{quotation}
Please do not use quotation marks when quoting texts! Simply use the \verb|quotation| environment -- it will automatically render Springer's preferred layout.
\end{quotation}


\subsubsection{Subsubsection Heading}
Instead of simply listing headings of different levels we recommend to
let every heading be followed by at least a short passage of text.
Further on please use the \LaTeX\ automatism for all your
cross-references and citations as has already been described in
Sect.~\ref{subsec:2}, see also Fig.~\ref{fig:1}\footnote{If you copy
text passages, figures, or tables from other works, you must obtain
\textit{permission} from the copyright holder (usually the original
publisher). Please enclose the signed permission with the manuscript. The
sources\index{permission to print} must be acknowledged either in the
captions, as footnotes or in a separate section of the book.}

Please note that the first line of text that follows a heading is not indented, whereas the first lines of all subsequent paragraphs are.

% For figures use
%
\begin{figure}[b]
\sidecaption
% Use the relevant command for your figure-insertion program
% to insert the figure file.
% For example, with the graphicx style use
%\includegraphics[scale=.65]{figure}
%
% If no graphics program available, insert a blank space i.e. use
%\picplace{5cm}{2cm} % Give the correct figure height and width in cm
%
\caption{If the width of the figure is less than 7.8 cm use the \texttt{sidecapion} command to flush the caption on the left side of the page. If the figure is positioned at the top of the page, align the sidecaption with the top of the figure -- to achieve this you simply need to use the optional argument \texttt{[t]} with the \texttt{sidecaption} command}
\label{fig:1}       % Give a unique label
\end{figure}


%
\begin{acknowledgement}
Acknowledge the NSF.  Wolfe.  Blumenthal
\end{acknowledgement}
%
%\section*{Appendix}
%\addcontentsline{toc}{section}{Appendix}
%
%
%When placed at the end of a chapter or contribution (as opposed to at the end of the book), the numbering of tables, figures, and equations in the appendix section continues on from that in the main text. Hence please \textit{do not} use the \verb|appendix| command when writing an appendix at the end of your chapter or contribution. If there is only one the appendix is designated ``Appendix'', or ``Appendix 1'', or ``Appendix 2'', etc. if there is more than one.

%\input{referenc}
% \bibliographystyle{}
% \bibliography{}
\end{document}
